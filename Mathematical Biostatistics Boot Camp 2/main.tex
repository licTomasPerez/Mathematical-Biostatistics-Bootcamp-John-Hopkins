\documentclass{homework}
\author{Tomás Pérez}
\class{Mathematical Biostatistics Boot Camp 1}
\date{\today}
\title{Homework Assignments and Quizzes}

\graphicspath{{./media/}}

\begin{document} \maketitle

\section{Homework 1 - Week 1}
\begin{tcolorbox}[title=Question 1]
In a  random sample of 100 patients at a clinic, you would like to test whether the mean RDI is 10 or more using a one sided 5\% type 1 error rate.

The sample mean RDI was 12 with a standard deviation of 4. Do you reject the null hypothesis?
\end{tcolorbox}

Here we have a one sided test for a type I error rate of $\alpha = .05$. Our null hypothesis is $H_0: \mu = \mu_0 = 10$, where our sample size is made up of 100 patients, a sample mean RDI was $12$ with a standard deviation of $4$. Our test statistic is $TS = \frac{\bar{X}-\mu_0}{S/\sqrt{n}}$, and the alternative hypothesis is $H_a : \mu > \mu_0 = 10$ for example. We'll reject in favour of the alternative hypothesis if $TS \geq z_{1-\alpha}$, where $z_{1-\alpha}$ is the $(1-\alpha)$th standard normal quantile. Then, we can compute this with the following R routine

\begin{lstlisting}[language=R]
n <- 100
mu <- 12
mu0 <- 10 
sd <- 4
alpha <- .05

testStat <- (mu - mu0)/(sd/sqrt(n))
qtile <- qnorm(alpha, lower.tail = FALSE)

## We reject the null hypothesis if the test statistic is greater or equal than the quantile
reject <- testStat > qtile 
print(reject)

In [1]: TRUE
\end{lstlisting}

Then, we reject the null hypothesis in favour of the alternative hypothesis. \\

\begin{tcolorbox}[title=Question 2]
A pharmaceutical company is interested in testing a potential blood pressure lowering medication. Their first examination considers only subjects that received the medication at baseline then two weeks later. The data are as follows (SBP in mmHg)

\begin{center}
\begin{tabular}{ |c|c|c|c| } 
\hline
Subject & Baseline & Week2 \\ [0.5ex] 
 \hline
1 & 140 & 132 \\ 
2 & 138 & 135 \\
3 & 150 & 151 \\
4 & 148 & 146 \\
5 & 135 & 130 \\ [1ex] 
 \hline
\end{tabular}
\end{center}

\end{tcolorbox}

We are dealing with a pair of two non-independent data groups, thus we are to perform a two-sided paired test. Let $H_0: \mu_d = 0$ versus $H_a: \mu_d \neq 0$ where $\mu_d$ is the mean difference between followup and baseline. Our statistic of interest for this \textbf{ordinary paired two-group T-test} is 

$$
\frac{\bar{X}_d - \mu_{d0}}{S_d/\sqrt{n}}.
$$

Then, with the following R routine, we can know weather to reject or fail to reject the null hypothesis.

\begin{lstlisting}[language=R]
group1 = c(140,138,150,148,135) # R-array of group 1-data points
group2 = c(132,135,151,146,130) # R-array of group 2-data points
diff <- group2-group1 # difference of the two data groups
n <- sum(!is.na(diff)) # number of pairs of data points
mu <- mean(diff) # mean of the difference of the data groups
standard_deviation <- sd(diff) # standard deviation of the data groups 
testStat <- sqrt(n) * (mean(diff) - 0)/sd(diff) 

Pvalue <- 2 * pt(abs(testStat), n-1, lower.tail = FALSE) 
Pvalue

In [1]: 0.08652278
------------

group1 = c(140,138,150,148,135) # R-array of group 1-data points
group2 = c(132,135,151,146,130) # R-array of group 2-data points
t.test(group2, group1, alternative="two.sided", paired=TRUE)
t.test(diff)

Paired t-test

In [1]: data:  group2 and group1
In [2]: t = -2.2616, df = 4, p-value = 0.08652
In [3]: alternative hypothesis: true difference in means is not equal to 0
In [4]: 95 percent confidence interval:
In [5]: -7.5739122  0.7739122
In [6]: sample estimates:
In [7]: mean of the differences 
In [8]:             -3.4 
\end{lstlisting}

Note the equivalence with a two-sided paired T-test for a single data group

\begin{lstlisting}[language=R]
group1 = c(140,138,150,148,135) # R-array of group 1-data points
group2 = c(132,135,151,146,130) # R-array of group 2-data points
t.test(group2 - group1, alternative = "two.sided")

Paired t-test

In [1]: data:  group2 and group1
In [2]: t = -2.2616, df = 4, p-value = 0.08652
In [3]: alternative hypothesis: true difference in means is not equal to 0
In [4]: 95 percent confidence interval:
In [5]: -7.5739122  0.7739122
In [6]: sample estimates:
In [7]: mean of the differences 
In [8]:             -3.4 
\end{lstlisting}

But this is not equivalent to a one-sided two-sided test:

\begin{lstlisting}[language=R]
group1 = c(140,138,150,148,135) # R-array of group 1-data points
group2 = c(132,135,151,146,130) # R-array of group 2-data points

In [1]: Paired t-test
data: group2 and group1
In [2]: t = -2.262, df = 4, In In [3]: p-value = 0.04326
In [3]: alternative hypothesis: In [4]: true difference in means is less than 0
In [5]: 95 percent confidence interval:
In [6]: -Inf -0.1951
In [7]: sample estimates:
In [8]: mean of the differences
In [9]:             -3.4
\end{lstlisting}

In conclusion, the P-value is equal to .09 and we fail to reject the null hypothesis. \\ 

\begin{tcolorbox}[title=Question 3]
A sample of 9 men yielded a sample average brain volume of 1,100cc and a standard deviation of 30cc. For what values of $\mu_0$ would a test of $H_0: \mu = \mu_0$  fail to reject the null hypothesis in a two sided 5\% Students t-test?
\end{tcolorbox}

Here we just need to construct the 95\% Student's $T$ confidence interval.

\begin{lstlisting}[language=R]
n <- 9
mu0 <- 1100
sd <- 30

CI <- (c(-1,1) * qt(0.975,n-1) * sd/sqrt(n-1) + mu0)
CI

In [1]: 1075.541 1124.459
\end{lstlisting}

Thus, we find the answer is $[1076cc, 1124 cc]$. 

\clearpage

\begin{tcolorbox}[title=Question 4]
To further test a hospital triage system, administrators selected 200 nights and randomly assigned a new triage system to be used on 100 nights and a standard system on the remaining 100 nights. They calculated the nightly median waiting time (MWT) to see a physician. The average MWT for the new system was 4 hours with a standard deviation of .5 hours while the average MWT for the old system was 6 hours with a standard deviation of 2 hours. Test the hypothesis of a decrease in the mean MWT associated with the new treatment using a two sided 5\% Z test. Does it appear to be effective?
\end{tcolorbox}

Let $H_0: \mu = 0$ be the null hypothesis (ie. there's no difference in response times between the hospital triage systems). In this case, we have to construct a 95\% Student's $T$-test for the difference of two random variables. Assuming our data are iid Gaussian and given that $n_1, n_2 \gg 1$ we can use the standard normal quantiles, then the statistic of interest is 

$$
\bar{Y}-\bar{X}\pm z_{1-\frac{\alpha}{2}} \sqrt{\frac{S_x^2}{n_X}+\frac{S_y^2}{n_Y}}.
$$

\begin{lstlisting}[language=R]
n_old <- n_new <- n_pairs <- 100
mu_old <- 4
sd_old <- .5 
n_new <- 100
mu_new <- 6
sd_new <- 2
test_stat <- mu_new - mu_old + c(-1, 1) * qnorm(0.975) * sqrt(sd_new^2/n_new + sd_old^2/n_old)
test_stat

In [1]: 1.595943 2.404057
\end{lstlisting}

Since this interval doesn't contain zero, we reject the null hypothesis and conclude that the new system does reduce MWTs. \\

\begin{tcolorbox}[title=Question 5]
Suppose that 18 obese subjects were randomized, 9 each, to a new diet pill and a placebo. Subjects’ body mass indices (BMIs) were measured at a baseline and again after having received the treatment or placebo for four weeks. The average difference from follow-up to the baseline (followup - baseline) was -3 kg/m2 for the treated group and 1 kg/m2 for the placebo group. The corresponding standard deviations of the differences was 1.5 kg/m2 for the treatment group and 1.8 kg/m2 for the placebo group. Does the change in BMI over the two year period appear to differ between the treated and placebo groups?  Assuming normality of the underlying data and a common population variance, give the relevant 95\% t confidence interval.
\end{tcolorbox}

Let $H_0: \mu_{\textnormal{diff, treated}} < \mu_{\textnormal{diff, placebo}}$ be the null hypothesis. We perform an ordinary two-sided paired $T$ test on the two data groups. Our statistic of interest is 

$$
\frac{\bar{X}_d - \mu_{d0}}{S_d/\sqrt{n}}.
$$

We can determine whether to reject or fail to reject the null hypothesis with the following R routine:

\begin{lstlisting}[language=R]
n1 <- n2 <- 9
x_t <- -3
sd_t <-1.5
x_p <- 1
sd_p <- 1.8
alpha <- 0.05
t_qt <- qt(1-alpha/2, n1+n2-2)

sp <- sqrt(((n1-1)*sd_t^2+(n2-1)*sd_p^2)/(n1+n2-2))
test_stat <- (x_t-x_p)/(sp*sqrt(1/n1+1/n2))

reject <- test_stat < -t_qt
reject

In [1]: TRUE
\end{lstlisting}

Thus, we reject at the 5\% level. \\

\begin{tcolorbox}[title=Question 6]
Suppose that 18 obese subjects were randomized, 9 each, to a new diet pill and a placebo. Subjects’ body mass indices (BMIs) were measured at a baseline and again after having received the treatment or placebo for four weeks. The average difference from follow-up to the baseline (followup - baseline) was -3 kg/m2 for the treated group and 1 kg/m2 for the placebo group. The corresponding standard deviations of the differences was 1.5 kg/m2 for the treatment group and 1.8 kg/m2 for the placebo group. Does the change in BMI over the two year period appear to differ between the treated and placebo groups?  Assuming normality of the underlying data and a common population variance, give the relevant 95\% t confidence interval.
\end{tcolorbox}

We need to construct the 95\%th $T$ confidence interval for the difference of two random variables. A $(1-\alpha)\times 100\%$-confidence interval for $\mu_Y-\mu_X$ can be written as follows

$$
\bar{Y}-\bar{X}\pm t_{n_{X}+n_{Y}-2,1-\frac{\alpha}{2}}S_p \sqrt{\frac{1}{n_X}+\frac{1}{n_Y}},
$$

where $S_p^2 = \frac{(n_X-1)S_x^2 + (n_Y-1)S_y^2 }{n_X+n_Y-2}$ is the pooled variance (an unbiased estimator for the variance $\sigma^2$). We can found this CI with the following R routine

\begin{lstlisting}[language=R]
n1 <- n2 <- 9
x_t <- -3
sd_t <-1.5
x_p <- 1
sd_p <- 1.8
alpha <- 0.05
t_qt <- qt(1-alpha/2, n1+n2-2)

sp <- sqrt(((n1-1)*sd_t^2+(n2-1)*sd_p^2)/(n1+n2-2))

CI <- (x_t - x_p) + c(-1,1) * t_qt * sp * sqrt(1/n1+1/n2)
CI

In [1]: -5.655699 -2.344301
\end{lstlisting}

Then the 95th confidence interval is $[-5.656, -2.344]$. \\

\begin{tcolorbox}[title=Question 7]
Suppose that systolic blood pressures were taken on 16 oral contraceptive users and 16 controls at baseline and again then two years later. The average difference from follow-up SBP to the baseline (followup - baseline) was 11 mmHg for oral contraceptive users and 4 mmHg for controls. The corresponding standard deviations of the differences was 20 mmHg for OC users and 28 mmHg for controls. Assuming that blood pressure are normally distributed and an equal variance, give a P-value for a two sided hypothesis test that the change in BP differs for OC users compared to controls.
\end{tcolorbox}

We need to find the P-value for a two sided hypothesis test with the following R routine: 

\begin{lstlisting}[language=R]
n1 <- n2 <- 16
dif1 <- 11
dif2 <- 4
sd_dif1 <- 20
sd_dif2 <- 28

sp <- sqrt((sd_dif1^2+sd_dif2^2)/2)
test_stat <- (dif2 - dif1)/(sp * sqrt(1/n1 + 1/n2))
p_value <- 2 * pt(test_stat, df = (n1+n2-2))
p_value

In [1]: 0.4222096
\end{lstlisting}

\begin{tcolorbox}[title=Question 8]
Researchers would like to conduct a study of 100 healthy adults to detect a four year mean brain volume loss of $.01~mm^3$. Assume that the standard deviation of four year volume loss in this population is $.04~mm^3$. What would be the power of the study for a 5\% one sided test versus a null hypothesis of no volume loss?
\end{tcolorbox}

\begin{lstlisting}[language=R]
no_sim <- 100000 # number of simulation to perform 
n_dof <- 100 # number of degrees of freedom
sigma <- 0.04 # variance 
alpha <- .05 

mu0 <- .01 # RDI mean under the null hypothesis
mua <- 0 # RDI mean under the alternative hypothesis
z <- rnorm(no_sim) # rnorm is the R function that simulates random variables having a specified normal distribution
chisq <- rchisq(no_sim, df = n_dof - 1) # chi squared distribution
t_qt <- qt(.95, n_dof-1) # 95th quantile for the Gossett's T distribution
mean(z + sqrt(n_dof)*(mua-mu0)/sigma > 
    t_qt/sqrt(n_dof-1)*sqrt(chisq))
    
In [1]: 0
\end{lstlisting}

\begin{tcolorbox}[title=Question 9]
The Daily Planet ran a recent story about Kryptonite poisoning in the water supply after a recent event in Metropolis. Their usual field reporter, Clark Kent, called in sick and so Lois Lane reported the stories. Researchers sampled 288 individuals and found mean blood Kryptonite levels of 44, both measured in Lex Luthors per milliliter (LL/ml). They compared this to 288 sampled individuals from Gotham city who had an average level of 42.04. Report the Pvalue for a two sided Z test of the relevant hypothesis. Assume that the standard deviation is 12 for both groups.
\end{tcolorbox}

Let $H_0 : \mu_{\textnormal{Metropolis}} = \mu_{\textnormal{Gotham}}$ be the null hypothesis versus the alternative hypothesis $H_a : \mu_{\textnormal{Metropolis}} \neq \mu_{\textnormal{Gotham}}$. We need to report the P-value for a two sided $Z$ test. Let our test statistic be

$$
TS = \frac{\bar{X}_{\textnormal{Metropolis}} - \bar{X}_{\textnormal{Gotham}} }{\sigma\sqrt{\frac{1}{n_{\textnormal{Metropolis}}} + \frac{1}{n_{\textnormal{Gotham}}}}},
$$

we'll reject this hypothesis if $|TS| \geq Z_{1-\alpha}$, where $Z_{1-\alpha} = 1.96$ is the 95th quantile for the standard normal distribution. Thus we want

\begin{align*}
\mathds{P}\bigg(|TS| \geq 1.96 \bigg| \mu_{\textnormal{Metropolis}} \neq \mu_{\textnormal{Gotham}} \bigg) &= \mathds{P}\bigg(\bigg|\frac{\bar{X}_{\textnormal{Metropolis}} - \bar{X}_{\textnormal{Gotham}} }{\sigma\sqrt{\frac{1}{n_{\textnormal{Metropolis}}} + \frac{1}{n_{\textnormal{Gotham}}}}}\bigg| \geq 1.96 \bigg| \mu_{\textnormal{Metropolis}} \neq \mu_{\textnormal{Gotham}}\bigg) \\
&= \mathds{P}\bigg(\bigg|\bar{X}_{\textnormal{M.}} - \bar{X}_{\textnormal{G.}}\bigg| \geq 1.96 \times \sigma\sqrt{\frac{1}{n_{\textnormal{M.}}} + \frac{1}{n_{\textnormal{G.}}}} \bigg| \mu_{\textnormal{M.}} \neq \mu_{\textnormal{G.}}\bigg)
\end{align*}

where this probability is calculated under the alternative hypothesis. Note that, under the null hypothesis, $\bar{X}_{\textnormal{M.}} - \bar{X}_{\textnormal{G.}} \sim \mathcal{N}(2,1)$. This P-value can be computed with the following R routine

\begin{lstlisting}[language=R]
P_value <- pnorm(-1.96, mean = 2, sd = 1) + pnorm(1.96, mean = 2, sd = 1)
P_value 

In [1]: 0.484084
\end{lstlisting}

Thus, the P-value is about $0.05$, as usual. \\

\begin{tcolorbox}[title=Question 10]
As your variance goes up, what happens to power?
\end{tcolorbox}

Higher variability implies more dispersion of the results. Less variability means more power; more variability means less power. \\

\begin{tcolorbox}[title=Question 11]
Suppose that you have three independent samples from a $N(\mu_1, \sigma^2)$, $N(\mu_2, \sigma^2)N$ and $N(\mu_3, \sigma^2)$ respectively of size $n_1$, $n_2$ and $n_3$. Let $S_1^2$, $S_2^2$ and $S_3^2$ be the associated sample variances. Would $\frac{1}{3}S_1^2 + \frac{1}{3} S_2^2 + \frac{1}{3} S_3^2$ be unbiased?
\end{tcolorbox}

Since the expectation value $\mathds{E}[\cdot]$ is a linear operator, the sum is always unbiased regardless of normality. \\

\begin{tcolorbox}[title=Question 12]
Consider any of the hypothesis tests we've covered in class. If you know you reject at a specific level of $\alpha$, then you know what else?
\end{tcolorbox}

All of the tests we've covered in class have the property that, if you reject for a given type one error rate, you will reject for any larger type one error rate. In fact, it would be strange for a test to not have this property. \\

\begin{tcolorbox}[title=Question 13]
Consider a one sided $\alpha$ level single group Z test of $H_0 : \mu = \mu_0$ versus the alternative $H_a : \mu < \mu_0$ with the data $\bar X$ for the sample mean and s for the sample standard deviation with n measurements. What are the collection of points for which you would fail to reject the hypothesis?
\end{tcolorbox}

For a one sided single group $Z$-test for a null hypothesis $H_0 : \mu = \mu_0$ versus the alternative $H_a : \mu < \mu_0$ has a fail-to-reject region if 

\begin{align*}
    Z \geq z_{\alpha} &\rightarrow \frac{\bar{X}-\mu_0}{s/\sqrt{n}} \geq z_{\alpha} \\
    &\rightarrow \bar{X} - z_{\alpha}\frac{s}{\sqrt{n}} \geq \mu_0 \\
    &\rightarrow  \bar{X} + z_{\alpha}\frac{s}{\sqrt{n}} \geq \mu_0
\end{align*}

which is the same as $(-\infty,\bar{X} + z_{\alpha}\frac{s}{\sqrt{n}})$. \\

\clearpage

\section{Quiz 1 - Week 1}

\begin{tcolorbox}[title=Question 1]
In a  random sample of 100 patients at a clinic, you would like to test whether the mean RDI is $x$ or more using a one sided 5\% type I error rate. The sample mean RDI had a mean of 12 and a standard deviation of 4. What value of $x$ (testing $H_0 : \mu = x$ versus $H_a : \mu > x$) would you reject for?
\end{tcolorbox}

Let the null hypothesis be $H_0 : \mu = x$ and let the alternative hypothesis be $H_a : \mu > x$, where the sample size is $n=100$, the sample mean RDI is $\mu_0$ is $12$ with a standard deviation of $4$ and a one-sided 5\% type I error rate. Our $Z$-test statistic of interest is 

$$
TS = \frac{x-\mu_0}{S/\sqrt{n}} = \frac{x-12}{4/\sqrt{100}}.
$$

In this type of one-sided test, we'll reject in favour of the alternative hypothesis if our sample mean (expressed in standard error units) is enough above $\mu_0$, with the threshold being given by the 95th standard normal quantile. This is 

\begin{align*}
    TS & \geq z_{1-\alpha} \Rightarrow \frac{x-\mu_0}{S/\sqrt{n}} \geq z_{1-\alpha} \\
    & \Rightarrow x -\mu_0 \geq z_{1-\alpha} \frac{S}{\sqrt{n}} \\
    & \Rightarrow x \geq \mu_ 0 +z_{1-\alpha} \frac{S}{\sqrt{n}} 
\end{align*}

Therefore, plugging in the numbers (given that $z_{95}=1.645$) we get that the rejection region is 

$$
\textnormal{Rejection Region}(x) = [12.66,\infty).
$$

ie. we reject in favour of the alternative hypothesis if $x \geq 12.66$

\begin{tcolorbox}[title=Question 2]
A pharmaceutical company is interested in testing a potential blood pressure lowering medication. Their first examination considers only subjects that received the medication at baseline then two weeks later. The data
is given as follows (SBP in mmHg)

\begin{center}
\begin{tabular}{ |c|c|c|c| } 
\hline
 Baseline & Week2 \\ [0.5ex] 
 \hline
 140 & 138 \\ 
 138 & 136 \\
 150 & 148 \\
 148 & 146 \\
 135 & 133 \\ [1ex] 
 \hline
\end{tabular}
\end{center}

Test the hypothesis that there was a mean reduction in blood pressure. Compare the difference between a paired and unpaired test for a two sided 5\% level test.
\end{tcolorbox}

In this case, we actually want to perform two tests. An ordinary paired two-group t-test and an independent two-group t-test. Since, in reality, the observations are paired (the data group 1 and data group 2 both consist of observations of the same test subjects), the correct way to do a statistical analysis would be with the ordinary paired two group t-test. \\

Let $X$ represent the mean reduction in blood pressure for the baseline group whilst $Y$ represents the mean reduction in blood pressure for week 2- test subjects and let $\mu_d = \mu_1-\mu_2$, the difference between the paired observation means.

\paragraph{\textbf{Ordinary Paired Two-group t-Test}} We'd like to test the null hypothesis $H_0: \mu_d = 0$ versus $H_a: \mu_d \neq 0$ (or one of the other two alternatives). Our test statistic of interest is 

$$\frac{X-Y}{S_d/\sqrt{n}}$$

where $S_d$ is the standard error, the sample size is $n=5$ and the type one error rate is $\alpha = .05$. This test can be performed with the following R-routine:

\begin{lstlisting}[language=R]
# Part A: one sided paired two-group ordinary test 
data_X_group1 <- c(140,138,150,148,135)
data_Y_group2 <- c(137,136,148,146,133)
alpha <- .05

diff <- data_X_group1 - data_Y_group2
n <- sum(!is.na(diff)) # number of pairs of data points
mu <- mean(diff) # mean of the difference of the data groups
std_error <- sd(diff) # standard deviation of the data groups 

testStat <- sqrt(n) * (mean(diff) - 0)/sd(diff) 

Pvalue <- 2 * pt(abs(testStat), n-1, lower.tail = FALSE) 

if(Pvalue > alpha){
    print("Reject the null hypothesis")
} else {
    print("Fail to reject the null hypothesis")
}

cat("Mean data group 1", mean(data_X_group1))
cat("Mean data group 2", mean(data_Y_group2))
cat("Mean difference", mean(diff))
cat("TS",testStat)
cat("P-value", Pvalue)

In [1]: "Fail to reject the null hypothesis"
In [2]: Mean data group 1 142.2
In [3]: Mean data group 2 140
In [4]: Mean difference 2.2
In [5]: TS 11
In [5]: P-value 0.0003881713
\end{lstlisting}

From this numerical calculation we gt the test statistic is $11$ and the P-value, the probability of obtaining a result as or more extreme than this test statistic, is $3\times 10^{-4}$. Since both $TS = 11 > z_{.975} = 1.96$ and $P < \alpha$, we reject the null hypothesis. \\

\paragraph{\textbf{Independent Two-group t-Test}}

Now, let $H_0 : \mu_1 = \mu_2$ versus $H_a: \mu_1 \neq \mu_2$ (or one of the other two alternatives). Assuming a common error variance we have that the following statistic

$$
\frac{\bar{X}-\bar{Y}}{S_p \sqrt{\frac{1}{n_X}+\frac{1}{n_Y}}} \sim t_{n_X+n_Y-2},
$$

where $S_p^2 = \frac{(n_X-1)S_x^2 + (n_Y-1)S_y^2 }{n_X+n_Y-2}$ is the pooled variance. Note that under the null hypothesis $\bar{X}-\bar{Y}$ has a hypothesised mean of $0$ if the data are iid Gaussian. We can perform this calculation with the following R-routine 

\begin{lstlisting}[language=R]
# Part B: one sided independent two-group ordinary test 
data_X_group1 <- c(140,138,150,148,135)
data_Y_group2 <- c(137,136,148,146,133)
alpha <- .05

mu1 <- mean(data_X_group1)
mu2 <- mean(data_Y_group2)
sd1 <- sd(data_X_group1)
sd2 <- sd(data_Y_group2)

n1 <- n2 <- 5 

sp <- sqrt(((n1-1)*sd1^2+(n2-1)*sd2^2)/(n1+n2-2)) # pooled variance
testStat <- (mu1 - mu2)/(sp * sqrt(1/n1+1/n2))

t_qt <- qt(1-alpha, n1+n2-2)

if(testStat > t_qt){
    print("Reject the null hypothesis")
} else {
    print("Fail to reject the null hypothesis")
}

cat("pooled variance", sp)
cat("test Statistic", testStat)
cat("95th T-quantile", t_qt)

In [1]: "Fail to reject the null hypothesis"
In [2]: pooled variance 6.545991
In [3]: test Statistic 0.5313948
In [4]: 95th T-quantile 1.859548
\end{lstlisting}

Thus, for an independent group test, we'd fail to reject the null hypothesis. Contrast this to the previous (the correct) result. \\

\begin{tcolorbox}[title=Question 3]
Brain volumes for 9 men yielded a 90\% confidence interval of 1,077 cc to 1,123 cc. Would you reject in a two sided 5\% hypothesis test of $H_0: \mu=1,078$? 
\end{tcolorbox}

\begin{tcolorbox}[title=Question 4]
In an effort to improve efficiency, hospital administrators are evaluating a new triage system for their emergency room. In an validation study of the system, 5 patients were tracked in a mock (simulated) ER under the new and old triage system. Let the waiting times in natural log of hours be:

\begin{center}
\begin{tabular}{ |c|c|c|c|c|c| } 
\hline
 Subject & 1 & 2 & 3 & 4 & 5 \\ [0.5ex]
 \hline
 New & 0.929 & -1.745 & 1.677 & 0.701 & 0.128 \\ 
 Old & 2.233 & -2.513 & 1.204 & 1.938 & 2.533 \\
 \hline
\end{tabular}
\end{center}

Give a P-value for the test of the hypothesis that the new system resulted in lower waiting times for a one sided t test.
\end{tcolorbox}

\begin{tcolorbox}[title=Question 5]
Refer to the previous question. Give a 95\% T confidence interval for the ratio of the waiting times (recall that the measurements were natural logged). The data is 

\begin{center}
\begin{tabular}{ |c|c|c|c|c|c| } 
\hline
 Subject & 1 & 2 & 3 & 4 & 5 \\ [0.5ex]
 \hline
 New & 0.929 & -1.745 & 1.677 & 0.701 & 0.128 \\ 
 Old & 2.233 & -2.513 & 1.204 & 1.938 & 2.533 \\
 \hline
\end{tabular}
\end{center}

In an effort to improve efficiency, hospital administrators are evaluating a new triage system for their emergency room. In an validation study of the system, 5 patients were tracked in a mock (simulated) ER under the new and old triage system. Their waiting times in natural log of hours were:
\end{tcolorbox}

\begin{tcolorbox}[title=Question 6]
Suppose that 18 obese subjects were randomized, 9 each, to a new diet pill and a placebo. Subjects’ body mass indices (BMIs) were measured at a baseline and again after having received the treatment or placebo for four weeks. The average difference from follow-up to the baseline (followup - baseline) was -3 kg/m2 for the treated group and 1 kg/m2 for the placebo group. The corresponding standard deviations of the differences was 1.5 kg/m2 for the treatment group and 1.8 kg/m2 for the placebo group. Does the change in BMI over the two year period appear to differ between the treated and placebo groups?  Assuming normality of the underlying data and a common population variance, give a pvalue for a two sided t test.
\end{tcolorbox}

\begin{tcolorbox}[title=Question 7]
Consider a one sided $\alpha$ level single group Z test of $H_0 : \mu = \mu_0$ versus $H_a : \mu > \mu_0$   with the data $\bar X$ for the sample mean and $s$ for the sample standard deviation with $n$ measurements. What are the collection of points for which you would fail to reject the hypothesis?
\end{tcolorbox}

\begin{tcolorbox}[title=Question 8]
Researchers would like to conduct a study of $n$ healthy adults to detect a four year mean brain volume loss of  $.01~mm^3$. Assume that the standard deviation of four year volume loss in this population is $.04~mm^3$. What would be the value of nn needed for $90\%$ power of type one error rate of $5\%$ one sided test versus a null hypothesis of no volume loss?
\end{tcolorbox}

\begin{tcolorbox}[title=Question 9]
The Daily Planet ran a recent story about Kryptonite poisoning in the water supply after a recent event in Metropolis. Their usual field reporter, Clark Kent, called in sick and so Lois Lane reported the stories. Researchers plan to sample 288 individuals from Metropolis and control city Gotham and will compare mean blood Kryptonite levels (in Lex Luthors per milliliter, LL/ml). The expect to find a mean difference in LL/ml of around 2. Assuming a two sided Z test of the relevant hypothesis at 5\%, what would be the power. Assume that the standard deviation is 12 for both groups.
\end{tcolorbox}

\begin{tcolorbox}[title=Question 10]
As you increase the type one error rate, $\alpha$, what happens to power?
\end{tcolorbox}

\begin{tcolorbox}[title=Question 11]
Consider a setting with iid data from a $\mathcal{N}(\mu, \sigma^2)$ distribution testing $H_0: \mu = 0$ versus $H_a : \mu > 0$. The null hypothesis is rejected if  $\sqrt{n}\bar X / \sigma > 1.645$. What happens to the type I error rate as $n$ goes to infinity?
\end{tcolorbox}

\begin{tcolorbox}[title=Question 12]
Suppose that you have three independent samples from a $\mathcal{N}(\mu_1, \sigma^2)$, $\mathcal{N}(\mu_2, \sigma^2)$ and $\mathcal{N}(\mu_3, \sigma^2)$ respectively of size $n_1$, $n_2$ and $n_3$. Let $S_1^2$, $S_2^2$ and $S_3^2$ be the associated sample variances. Define the pooled variance as

$$
S_p^2 = \frac{(n_1 - 1) S_1^2 + (n_2 - 1) S_2^2 + (n_3 - 1) S_3^2}{n_1 + n_2 + n_3 - 3}
$$

Consider testing $H_0: a \mu_1 + b \mu_2 + c \mu_3 = 0$. Let

$$
TS = \frac{a \bar X_1 + b \bar X_2 + c \bar X_3}{S_p \left( \frac{a^2}{n_1} + \frac{b^2}{n_2} + \frac{c^2}{n_3} \right) ^ { 1/ 2}}
$$

What distribution do you think $TS$ has under $H_0$?
\end{tcolorbox}

\begin{tcolorbox}[title=Question 13]
Consider a one sample Z test of $H_0 : \mu = \mu_0$ versus $H_a : \mu > \mu_0$. All else equal, which scenarios will be closer to rejecting the null hypothesis?
\end{tcolorbox}

\clearpage

\section{Homework 2 - Week 2}

\begin{tcolorbox}[title=Question 1]
Researchers are interested in estimating the natural log of the proportion of people in the population with hypertension. In a random sample of $n$ subjects, let ${X}$ be the number with hypertension. 

Create a confidence interval for the natural log of the proportion of people with hypertension. Assume that $n$ is very large.
\end{tcolorbox}

\begin{tcolorbox}[title=Question 2]
If the odds are 4 to 1 (i.e., just 4) of the NFL Ravens losing, what is the implied probability that they win? 
\end{tcolorbox}

\begin{tcolorbox}[title=Question 3]
In a randomly sampled survey of self-reported stress levels from two occupations, the following data were obtained where there were 100 in each group 
\end{tcolorbox}

\begin{tcolorbox}[title=Question 4]
Consider the following data recording case status relative to an environmental exposure


What would be the estimated asymptotic standard error for the log odds ratio for this data?
\end{tcolorbox}

\begin{tcolorbox}[title=Question 5]
If $x \sim \textnormal{Binomial}(n, p)$ and a Beta(2, 2) prior is placed on $p$, what is the posterior mean for $p$?
\end{tcolorbox}

\begin{tcolorbox}[title=Question 6]
Researchers conducted a blind taste test of Coke versus Pepsi. Each of four people was asked which of two blinded drinks given in random order they preferred. The data was such that 3 of the 4 people chose Coke. Assuming that this sample is representative, report a P-value for a test of the hypothesis that Coke is preferred to Pepsi using a one-sided exact test.
\end{tcolorbox}
\clearpage

\section{Quiz 2 - Week 2}

\begin{tcolorbox}[title=Question 1]
What is the delta method asymptotic standard error of $\sqrt{\hat p}$ where $\hat p$ is $X/n$ where $X \sim \textnormal{Binomial}(n, p)$
\end{tcolorbox}

\begin{tcolorbox}[title=Question 1]
You are a bookie taking \$1 bets on the NFL Ravens game. The odds you give betters determines their payout. If you give 4 to 1 odds against the Ravens winning, then if a person bets on the Ravens winning and they win, you owe them their original dollar, plus an additional \$4 (a total of \$5). If a person bets on the Ravens losing and they lose, then you owe them their original dollar back, plus \$0.25 (a total of \$1.25).

Suppose you collect a > 0 one dollar bets on the Ravens winning and b > 0 one dollar bets on the Ravens losing. What should you set the odds so that, regardless of the outcome of the game, you neither win nor lose money? Note, in this case, the betters place their bets and learn the odds later. Note also, the odds are something that you set in this case, not a direct measure of probability or randomness.
\end{tcolorbox}

$b/a$

\begin{tcolorbox}[title=Question 3]
In a randomly sampled survey of self-reported stress levels from two occupations, the following data were obtained


What is the P-value for a Z score test of equality of the proportions of high stress? (Note the notation 1e-5, for example, is $1\times 10^{-5}$)
\end{tcolorbox}

\begin{lstlisting}[language=R]
X <- 70
n1 <- 100
Y <- 15
n2 <- 100

p1 <- X / n1
p2 <- Y / n2

p.hat <- (X+Y)/(n1+n2)
test.statistics.null <- (p1-p2)/ ( sqrt( p.hat * ( 1 - p.hat ) * ( 1 / n1 + 1 / n2  )  )   )

# Two sided test
2 * pnorm(test.statistics.null, lower.tail = FALSE)

In [1]: 3.627132e-15
\end{lstlisting}

\begin{tcolorbox}[title=Question 4]
Consider the following data recording case status relative to an environmental exposure


What would be the estimated asymptotic standard error for the log relative risk for this data? Consider case status as the outcome. That is, we're interested in evaluating the ratio of the proportion of cases comparing exposed to unexposed groups. 
\end{tcolorbox}

\begin{lstlisting}[language=R]
n1 <- 45+21
p1 <- 45/n1

n2 <- 15+52
p2 <- 15/n2

RR <- p1 / p2

RR.log <- log(RR) 

Standard.Error.Log.RR <- sqrt( ((1-p1)/(p1*n1)) + ((1-p2)/(p2*n2))   )
# 0.2425119

Standard.Error.RR <- exp(Standard.Error.Log.RR) 
Standard.Error.RR
\end{lstlisting}

\begin{tcolorbox}[title=Question 5]
Question 5
If $x_1 \sim \textnormal{Binomial}(n_1, p_1)$ and $x_2 \sim Binomial(n_2, p_2)$ and independent Beta(2, 2) priors are placed on $p_1$ and $p_2$, what is the posterior mean for $p_1 - p_2$?
\end{tcolorbox}

la resta 

\begin{tcolorbox}[title=Question 6]
Researchers conducted a blind taste test of Coke versus Pepsi. Each of four people was asked which of two blinded drinks given in random order that they preferred. The data was such that 3 of the 4 people chose Coke. Assuming that this sample is representative, report a P-value for a test of the hypothesis that Coke is preferred to Pepsi using a **two sided** exact test.
\end{tcolorbox}

\begin{lstlisting}[language=R]
binom.test(x=3, n = 4, p = 0.5, alternative = c("two.sided") )
\end{lstlisting}

\clearpage

\section{Homework 3 - Week 3}

\section{Quiz 3 - Week 3}

\end{document}
